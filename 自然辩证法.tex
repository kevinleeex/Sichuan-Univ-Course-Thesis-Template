\documentclass[a4paper,12pt]{report}
\usepackage{ctex}
\usepackage{xeCJK}
\usepackage{times}
\usepackage{setspace}
\usepackage{fancyhdr}
\usepackage{graphicx}
\usepackage{subfigure}
\usepackage{enumerate}
\usepackage{caption}
\usepackage{wrapfig}
\usepackage{array}  
\usepackage{fontspec,xunicode,xltxtra}
\usepackage{titlesec}
\usepackage{titletoc}
\usepackage[titletoc]{appendix}
\usepackage[top=30mm,bottom=30mm,left=20mm,right=20mm]{geometry}
\usepackage{cite}
\usepackage{listings}
\usepackage[framed,numbered,autolinebreaks,useliterate]{mcode}
\XeTeXlinebreaklocale "zh"
\XeTeXlinebreakskip = 0pt plus 1pt minus 0.1pt

\fancypagestyle{plain}{
	\pagestyle{fancy} 
}
%---------------------------------------------------------------------
%	页眉页脚设置
%---------------------------------------------------------------------
\pagestyle{fancy}
\lhead{\kaishu~自然辩证法课程期末报告~}
\rhead{\kaishu~20172230xxxxx~Kevin T. Lee~}
\cfoot{\thepage}

%---------------------------------------------------------------------
%	章节标题设置
%---------------------------------------------------------------------
\titleformat{\chapter}{\centering\zihao{-1}\heiti}{\chinese{chapter}}{1em}{}
\titlespacing{\chapter}{0pt}{*0}{*6}

%---------------------------------------------------------------------
%	摘要标题设置
%---------------------------------------------------------------------
\renewcommand{\abstractname}{\zihao{-3} 摘\quad 要}

%---------------------------------------------------------------------
%	参考文献设置
%---------------------------------------------------------------------
\renewcommand{\bibname}{\zihao{2}{\hspace{\fill}参\hspace{0.5em}考\hspace{0.5em}文\hspace{0.5em}献\hspace{\fill}}}

%---------------------------------------------------------------------
%	引用文献设置为上标
%---------------------------------------------------------------------
\makeatletter
\def\@cite#1#2{\textsuperscript{[{#1\if@tempswa , #2\fi}]}}
\makeatother

%---------------------------------------------------------------------
%	目录页设置
%---------------------------------------------------------------------
\titlecontents{chapter}[0em]{\songti\zihao{-4}}{\thecontentslabel\ }{}
{\hspace{.5em}\titlerule*[4pt]{$\cdot$}\contentspage}
\titlecontents{section}[2em]{\vspace{0.1\baselineskip}\songti\zihao{-4}}{\thecontentslabel\ }{}
{\hspace{.5em}\titlerule*[4pt]{$\cdot$}\contentspage}
\titlecontents{subsection}[4em]{\vspace{0.1\baselineskip}\songti\zihao{-4}}{\thecontentslabel\ }{}
{\hspace{.5em}\titlerule*[4pt]{$\cdot$}\contentspage}


\begin{document}
%---------------------------------------------------------------------
%	封面设置
%---------------------------------------------------------------------
\begin{titlepage}
	\begin{center}
		
    \includegraphics[width=0.9\textwidth]{figure//logo.jpg}\\
    \vspace{10mm}
    \textbf{\zihao{2}\kaishu{对自然辨证法与计算机科学发展关系的浅析}}\\[0.8cm]
    \textbf{\zihao{2}\kaishu{A touch to the relation between dialectics of nature and development of  computer science}}\\[3cm]
    
	\vspace{\fill}
	
\setlength{\extrarowheight}{3mm}
{\songti\zihao{3}	
\begin{tabular}{rl}
	
	{\makebox[4\ccwd][s]{学\qquad 院:}}& ~\kaishu 计算机学院\\
	{\makebox[4\ccwd][s]{课\qquad 程:}}& ~\kaishu 自然辩证法\\
	{\makebox[4\ccwd][s]{教\qquad 师:}}& ~\kaishu Peppa and George\\
	{\makebox[4\ccwd][s]{姓\qquad 名:}}& ~\kaishu Kevin T. Lee \\ 

    {\makebox[4\ccwd][s]{学\qquad 号:}}& ~\kaishu 20172230xxxxx \\ 

\end{tabular}
 }\\[2cm]
\vspace{\fill}
\zihao{4}

%---------------------------------------------------------------------
%  	日期
%---------------------------------------------------------------------
2017\textasciitilde 2018 第二学期\\
日期:2018年 6月28日
	\end{center}
\end{titlepage}

%---------------------------------------------------------------------
%  	摘要页
%---------------------------------------------------------------------
\begin{abstract}
\begin{spacing}{1.5}
	{\zihao{-4}
	随着计算机科学的发展与进步,计算机已经广泛地应用在社会的各个领域中,随着全球推动第三次工业革命的发展,各个国家着力发展计算机科学技术,构建信息化、智能化社会,人工智能作为计算机先进技术的代表之一,在现代及以后社会产生了或即将产生巨大的影响。自然辨证法是马克思主义对于自然界和科学技术发展以及人类认识自然改造自然的一般方法和科学,是辩证唯物主义的自然观、科学技术观和科学技术方法论。计算机科学的发展,离不开各种科学理论、方法的指导,自然辨证法正是其中关键的一环。本文主要基于自然辨证法的内涵针对计算机科学的几个方面进行论述,运用自然辨证法的基本理论、基本方法,浅析了自然辨证法与计算机科学技术的内在关系,阐述了自然辨证法对计算机科学技术发展的积极作用。\\[0.5cm]
	\textbf{关键字}:\quad 自然辨证法 \quad 计算机科学技术 \quad 人工智能 \quad 科学理论 \quad 
	}
\end{spacing}
\end{abstract}

%---------------------------------------------------------------------
%  	目录页
%---------------------------------------------------------------------
\tableofcontents % 生成目录

%---------------------------------------------------------------------
%  	自然辨证法简介
%---------------------------------------------------------------------
\chapter{自然辨证法简介}
\setcounter{page}{1}
\begin{spacing}{1.5}
\songti\zihao{-4}

	\section{含义}
	自然辨证法,是马克思主义对于自然界和科学技术发展的一般规律以及人类认识自然、改造自然的一般科学方法,是一门和形而上学相对立的、关于联系的科学。自然辩证法是马克思主义和恩格斯思想的自然观和自然科学观的反映,体现了马克思主义哲学和恩格斯思想的世界观、认识论、方法论的统一,构成马克思主义哲学的一个组成部分\cite{zrbzf}。
	
	马克思、恩格斯全面地系统地概括了他们所处时代的科学技术成功,批判吸取了前人的合理成分,系统地论述了辩证唯物主义的自然观、科学技术科学技术方法论。自然辨证法从人类发展与自然的关系出发,从马克思主义的的哲学高度上视察了自然界人类的实践活动、科技发展。自然界、人的认识与实践、科学技术由此成为自然辨证法三个要素。
	
	辩证法的规律是从自然界和人类社会的历史中抽象出来的,是历史发展的这两个方面和思维本身的最一般的规律。可以归结为下面三个规律:
	\begin{enumerate}[(1)]
	\item 质和量转化的规律
	\item 对立的相互渗透的规律
	\item 否定的否定的规律
	\end{enumerate}
	
	\section{意义}
	在科学技术观方面,马克思和恩格斯深刻的揭示了科技自身发展的内在逻辑,考察了作为一种社会现象的科技的发展。自然辨证法对人类社会发展和科技进步具有重大意义。经历过第一次工业革命和第二次工业革命后,人类逐渐进入信息化时代,开始第三次工业革命——信息工业革命,科技与社会的联系越来越紧密。科技的发展,社会的进步,需要对它们的内在规律进行研究。因此,自然辨证法在这个过程中充当了重要角色,发挥了重要作用\cite{rq2013}。
	
\end{spacing}

\chapter{与计算机科学技术的联系}
	\begin{spacing}{1.5}
	\songti\zihao{-4}
	
	\section{计算机科学技术的科学理论}
	进入信息时代的二十一世纪,计算机科学技术和信息技术蓬勃发展,从根本上改变了人们的工作、学习、生活的方式。工作中,各企业通过计算机科学技术的引入,是的生产效率大大提高,生产成本反而降低,更有一群互联网企业因此得利,成为行业中的领先者,带来了巨大的社会效应;在学习中,多媒体技术的广泛使用,提高了课堂效率,互联网创造了每个人公平的获取知识的机会;在生活中,新型的娱乐方式,支付方式,购物方式影响和改变了许多人的生活习惯。计算机科学技术已经成为我们生活中不可缺少的一部分,其重要性显而易见。
	
	随着当代科学技术的发展,不同学科之间相互渗透,交叉和综合已成为当今科技发展的一个重要趋势。许多社会上科学上的问题,往往都是需要不同学科之间的相互交叉和渗透。比如说最近正火热的人工智能,它正是一门具有高度综合性的前沿学科,综合了哲学、生理学、计算机应用学以及其他的自然学科。其核心是对人类大脑的模拟,本质是延长和增强人脑的智能,提高主体认知能力,是人类科学和社会实践活动中不可缺少的工具\cite{ai}。
	
	科学理论是关于对象领域本质及规律性的条理化、系统化了的理论知识体系。是被实践证明了的科学家说,其内容是人们借助抽象思维把握的关于事物本质和规律性的知识,其逻辑形式是概念、判断、推理及由此而组成的理论体系。
	
	建立科学理论体系有多种方法,其中从抽象上升到具体的方法、功利化方法、逻辑和历史相统一的方法是几种常用的方法。而由于计算机科学的特殊性,从抽象上升到具体是形成计算机科学理论的主要方法。
	
	比如软件工程理论体系的建立过程,一开始构成逻辑起点的概念是比较简单的、抽象的和贫乏的,而随着逻辑展开,概念的规定越来越复杂、具体和丰富,最后把软件制作各个环节的联系在思维中完整复制出来,即把事物作为整体在思维中再现出来,这种从抽象上升到具体的过程充分体现了自然辩证法的科学理论体系的构造过程。
	
	\section{自然辨证法和计算机科学的层次关系}
	我国科学家钱学森具体将学科分为四个层次,由高到低分别是哲学层、科学层、技术层、应用层。哲学层充分体现了自然辨证法的方方面面,从自然哲学的角度对人类社会发展的诸多宏观问题进行了研究;科学层体现了科学发展的各个方面,包括各种各样的微观算法,程序构建等。由此可见,自然辨证法和计算机科学分别着重体现了学科体系的两个层次,二者间存在着紧密的联系。正如工程问题上会涉及到伦理问题,计算机科学发展领域中涉及到的众多问题中,大多需要哲学学科为其进行一定分析,因此,计算机科学的发展需要一定自然辨证法的指导。
	
	\section{计算机科学技术中的自然辨证法原理}
	问题源于矛盾,大量的科学问题源于人类生活在世界上有多种多样的社会需要,当现有的生产技术手段或者科学理论不能满足某些社会需要时,就产生了尖锐的矛盾。比如计算机科学的产生就源于人类对高精度、高速度计算的追求与当时计算速度和精度低下的矛盾。
	
	计算机科学本身体现了多个学科的交叉性,人类面临的许多在经济、科技、社会问题,往往单门学科都难以应对只有综合应用自然科学和人文社会科学的知识和先进的技术,才能形成问题的最佳解决方案。学科的交叉、渗透、融合和创新,是科学综合和分化趋势的重要特征,也是学科发展的必然趋势。在21世纪,怎样培养出具备多方面学识修养、广阔的科学事业的复合型人才,是一项具有挑战性的重要课题。
	
	系统科学是把对象作为组织性、复杂性系统从整体上探索其存在方式和运动变化规律的学问,是对系统本质的正确反映和真理性认识。运用系统科学方法遵循的原则主要有:整体性原则,动态性原则,最优化原则,模块化原则\cite{sjkfzxjs}。
	
	计算机软件开发过程中,整体化原则是重要的原则。在需求分析阶段,有自顶向下和自底向上两种分析的方法,这两种方法各有优缺点\cite{sjkfzxjs},这是自然辨证法中提到的分析和综合的逻辑方法。
	
	人工智能是科学技术不断发展进步后的结果,人工智能是人类意识智能化的产物。人类是认识的主体,必然是社会的主体\cite{zqw2014}。人工智能正是伴随着人类实践和认识的进一步加深而产生的,人工智能的范围和内容也不断得到扩大和丰富。人工智能作为人类认识的客体,主体对客体的认识总是受客体发展程度的制约,所以人类智能的认识也依赖与人工智能的发展。
	
	计算机问题中的图灵测试是人工智能领域的一个经典测试,计算机和人脑的思维上有很大的不同,这是自然科学对于哲学层次问题的研究。自然辨证法对自然科学研究很有指导作用,如果说自然辩证法已经明白了机器和人脑思维的差异,那么对于人工智能的研究方向就有指导意义。
	
	\end{spacing}

\chapter{总结}
	\begin{spacing}{1.5}
	\songti\zihao{-4}
	社会的进步离不开人们对科学的不懈追求,时代的发展离不开人们利用辩证的思想去不停发现探索,运用科学的逻辑思维来思考问题,认识社会,把握事物的未来发展方向以及趋势。
	
	同时,科学在进步的同时也带来了诸多问题,我们要充分应用自然辩证法来分析科学技术的发展,遵循科学技术的发展规律,总结过去,把握当下,放眼未来,正确抉择计算机科学的发展方向。
	
	人工智能是增强人脑的智能,是人类智能意识化的产物,提高人类的认识能力,可以促进人工智能的发展,而人工智能作为人类认识的客体,主体对客体的认识总是受到客体发展程度的制约,所以人类智能的认识也依赖于人类智能的发展。
	
	在处理解决计算机科学技术发展过程中遇到的社会交叉问题时,自然辨证法科学理论和方法论可以给与我们启发,以及提供相应的理论基础和指导,这需要我们对自然辨证法有充分的认识,才能真正发挥好自然辨证法在计算机科学技术中的作用,两者才能更好的相辅相成。
	\end{spacing}

\chapter{致谢}
	\begin{spacing}{1.5}
	\songti\zihao{-4}
	感谢xxx老师和助教老师在课堂上的认真讲解和辛勤工作,为我们奉献了一节节生动有趣的课程,让我在课程中进一步学会了辨证看待问题的思想。同时谢谢各个小组同学悉心准备的报告,使我丰富了知识面,并进行了思想上的碰撞与交流。
	\end{spacing}

%---------------------------------------------------------------------
%  	参考文献设置
%---------------------------------------------------------------------
\addcontentsline{toc}{chapter}{参考文献}

\bibliographystyle{ieeetr}
\bibliography{refers}

		

\end{document}